\begin{tabularx}{\textwidth}{| r | X |}
\hline
\multicolumn{2}{|X|}{
\textbf{Use Case}: Mantener clientes} \\

\hline
\multicolumn{2}{|c|}{\cellcolor[gray]{0.6}} \\

\hline
\multicolumn{2}{|X|}{
\textbf{Descripción}: Mantenimiento de la información relacionada a los
clientes.} \\

\hline
\multicolumn{2}{|X|}{
\textbf{Actores participantes}: Administrador} \\

\hline
\multicolumn{2}{|c|}{\cellcolor[gray]{0.6} } \\

\hline
\multicolumn{2}{|X|}{
\textbf{Flujos}} \\

\hline
\multicolumn{2}{|X|}{
\textbf{Flujo principal}} \\

\hline
1 & El sistema solicita el DNI, nombre, apellido o teléfono del cliente. \\
\hline
2 & El administrador proporciona cualquier combinación de los campos
solicitados (por ejemplo, proporcionando nombre y apellido, o sólo nombre, o
teléfono y DNI, etc.). \\
\hline
3 & El sistema valida que se haya proporcionado información en al menos uno de
los campos (E1 si no se proporcionó ninguna información). \\
\hline
4 & El sistema realiza una búsqueda de todos los clientes registrados cuya
información coincida, aunque sea parcialmente, con los datos proporcionados. \\
\hline
5 & El sistema muestra un listado con todos los clientes cuya información
coincide, al menos parcialmente, con la información proporcionada. Dicho
listado debe incluir DNI, nombre, apellido y teléfono del cliente. Por cada
cliente, el sistema proporciona la opción de modificar dicha información. (S1
si el administrador elige la opción de modificar el cliente). \\
\hline
6 & Fin del caso de uso. \\

\hline
\multicolumn{2}{|X|}{
\textbf{Flujos alternativos}} \\

\hline
S1.1 & El sistema solicita el nombre, apellido, DNI, límite de compra,
domicilio, teléfono y número de tarjeta, completando por defecto cada dato con
el valor almacenado actualmente en el sistema para el cliente. \\
\hline
S2.1 & El sistema almacena los nuevos valores de cada campo para el cliente. \\
\hline
S1.2 & El caso de uso continua en 6. \\

\hline
\multicolumn{2}{|X|}{
\textbf{Flujos de excepción}} \\

\hline
1.1 & El sistema muestra un mensaje de error indicando que debe completarse al
menos uno de los criterios de búsqueda. \\
\hline
1.2 & El caso de uso continua en 6. \\


\hline
\end{tabularx}

