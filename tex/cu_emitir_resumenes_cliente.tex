\begin{tabularx}{\textwidth}{| r | X |}
\hline
\multicolumn{2}{|X|}{
\textbf{Use Case}: Emitir resúmen Cliente} \\

\hline
\multicolumn{2}{|c|}{\cellcolor[gray]{0.6}} \\

\hline
\multicolumn{2}{|X|}{
\textbf{Descripción}: Cálculo de intereses mensuales y emisión de
resúmenes asociados.} \\

\hline
\multicolumn{2}{|X|}{
\textbf{Actores participantes}: Mensualmente, Cliente, Comercio} \\

\hline
\multicolumn{2}{|c|}{\cellcolor[gray]{0.6} } \\

\hline
\multicolumn{2}{|X|}{
\textbf{Flujos}} \\

\hline
\multicolumn{2}{|X|}{
\textbf{Flujo principal}} \\

\hline
1 & El sistema llama a ``Informar Movimientos Mensuales'' procesando las compras realizadas por cada cliente en los distintos
comercios. \\

\hline
2 & Por cada cliente: \\
\hline
2.1 & El sistema obtiene el interés por deuda acumulada calculando el 5\% del
saldo de la cuenta del cliente. \\
\hline
2.2 & El sistema obtiene todos los movimientos informados para el cliente
fechados en el mes anterior al mes actual. \\
\hline
2.3 & El sistema obtiene el nuevo saldo de la cuenta del cliente sumando al
saldo adeudado el monto por interés obtenido en 2.1 más la suma de todos los
importes de los movimientos obtenidos en 2.2. \\
\hline
2.4 & El sistema emite un informe impreso que detalla el saldo adeudado
anteriormente, el interés calculado en 2.1, los movimientos obtenidos en 2.2 y
el nuevo total adeudado calculado en 2.3, siempre y cuando no se de
simultaneamente que el nuevo saldo total adeudado sea inferior a un centavo
(0.01) y que la fecha de expiración de la tarjeta sea anterior a la fecha
actual.\\
\hline
2.5 & El sistema actualiza el saldo de la cuenta del cliente al valor obtenido en 2.3. \\
\hline
3 & Fin del caso de uso. \\

\hline
\end{tabularx}

