\begin{tabularx}{\textwidth}{| r | X |}
\hline
\multicolumn{2}{|X|}{
\textbf{Use Case}: Informar ventas} \\

\hline
\multicolumn{2}{|c|}{\cellcolor[gray]{0.6}} \\

\hline
\multicolumn{2}{|X|}{
\textbf{Descripción}: Registración de las compras sumarizadas por día en un
comercio particular.} \\

\hline
\multicolumn{2}{|X|}{
\textbf{Actores participantes}: Comercio} \\

\hline
\multicolumn{2}{|c|}{\cellcolor[gray]{0.6} } \\

\hline
\multicolumn{2}{|X|}{
\textbf{Flujos}} \\

\hline
\multicolumn{2}{|X|}{
\textbf{Flujo principal}} \\

\hline
1 & El comercio informa las compras realizadas por los distintos clientes en un
día particular, proporcionando la fecha para la cual se están informando las
compras, el código del comercio que las está informando y un listado que indica
el número de tarjeta y el importe sumarizado de todas las compras en la fecha
informada para dicha tarjeta.\\
\hline
2 & El realiza las siguientes validaciones (E1 si alguna de ellas fallase). 
\begin{enumerate}
\item Debe existir un comercio con el código de comercio informado.
\item La fecha proporcionada debe corresponder al mes actual.
\item Los números de tarjeta proporcionados deben estar todos asociados cada
uno a una tarjeta activa.
\end{enumerate}
\\
\hline
3 & El sistema registra las transacciones informadas. \\
\hline
4 & Fin del caso de uso. \\

\hline
\multicolumn{2}{|X|}{
\textbf{Flujos de excepción}} \\

\hline
E1.1 & El sistema emite un mensaje de error informando que los datos informados
son incorrectos, adjuntando un mensaje que indica la validación que falló. \\
\hline
E1.2 & El caso de uso continua en 4. \\

\hline
\end{tabularx}

