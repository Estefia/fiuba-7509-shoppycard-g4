\begin{tabularx}{\textwidth}{| r | X |}
\hline
\multicolumn{2}{|X|}{
\textbf{Use Case}: Cancelar renovación} \\

\hline
\multicolumn{2}{|c|}{\cellcolor[gray]{0.6}} \\

\hline
\multicolumn{2}{|X|}{
\textbf{Descripción}: Cancela el proceso de renovación automática de la tarjeta,
de manera que la tarjeta sigue vigente hasta que expire. A partir de su
expiración, la tarjeta se mantiene como inactiva y no puede ser utilizada.} \\

\hline
\multicolumn{2}{|X|}{
\textbf{Actores participantes}: Cliente} \\

\hline
\multicolumn{2}{|c|}{\cellcolor[gray]{0.6} } \\

\hline
\multicolumn{2}{|X|}{
\textbf{Flujos}} \\

\hline
\multicolumn{2}{|X|}{
\textbf{Flujo principal}} \\

\hline
1 & El cliente solicita la cancelación de la renovación automática de su
tarjeta. \\
\hline
2 & El sistema solicita el número de tarjeta del cliente. \\
\hline
3 & El sistema valida que el número de tarjeta del cliente exista (E1.1 si no
existe). \\
\hline
4 & El sistema valida que la tarjeta del cliente esté activa y su renovación no
haya sido cancelada (E2.1 si está inactiva o su renovación ya ha sido
cancelada). \\
\hline
5 & El sistema marca la tarjeta como "no renovar". Se mantiene activa, pero en
su fecha de expiración la tarjeta no se renovará automáticamente. \\
\hline
6 & Fin del caso de uso. \\

\hline
\multicolumn{2}{|X|}{
\textbf{Flujos de excepción}} \\

\hline
E1.1 & El sistema emite un mensaje de error informando que no existe una tarjeta
con el número de tarjeta ingresado \\
\hline
E1.2 & El caso de uso continua en 6 \\

\hline
E2.1 & El sistema emite un mensaje de error informando que la tarjeta está
inactiva o que su renovación ya ha sido cancelada. \\
\hline
E2.2 & El caso de uso continua en 6 \\

\hline
\end{tabularx}

