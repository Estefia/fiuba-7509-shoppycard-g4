\begin{tabularx}{\textwidth}{| r | X |}
\hline
\multicolumn{2}{|X|}{
\textbf{Use Case}: Emitir resúmenes} \\

\hline
\multicolumn{2}{|c|}{\cellcolor[gray]{0.6}} \\

\hline
\multicolumn{2}{|X|}{
\textbf{Descripción}: Cálculo de intereses y comisiones mensuales y emisión de
resúmenes asociados.} \\

\hline
\multicolumn{2}{|X|}{
\textbf{Actores participantes}: Mensualmente, Cliente, Comercio} \\

\hline
\multicolumn{2}{|c|}{\cellcolor[gray]{0.6} } \\

\hline
\multicolumn{2}{|X|}{
\textbf{Flujos}} \\

\hline
\multicolumn{2}{|X|}{
\textbf{Flujo principal}} \\

\hline
1 & El sistema procesa las compras realizadas por cada cliente en los distintos
comercios. \\
\hline
2 & Por cada cliente: \\
\hline
2.1 & El sistema obtiene el interés por deuda acumulada calculando el 5\% del
saldo de la cuenta del cliente. \\
\hline
2.2 & El sistema obtiene todos los movimientos informados para el cliente
fechados en el mes anterior al mes actual. \\
\hline
2.3 & El sistema obtiene el nuevo saldo de la cuenta del cliente sumando al
saldo adeudado el monto por interés obtenido en 2.1 más la suma de todos los
importes de los movimientos obtenidos en 2.2. \\
\hline
2.4 & El sistema emite un informe impreso que detalla el saldo adeudado
anteriormente, el interés calculado en 2.1, los movimientos obtenidos en 2.2 y
el nuevo total adeudado calculado en 2.3, siempre y cuando no se de
simultaneamente que el nuevo saldo total adeudado sea inferior a un centavo
(0.01) y que la fecha de expiración de la tarjeta sea anterior a la fecha
actual.\\
\hline
2.5 & El sistema actualiza el saldo de la cuenta del cliente al valor obtenido en 2.3. \\
\hline
3 & Por cada comercio: \\
\hline
3.1 & El sistema determina todas las compras informadas por el comercio
fechadas al mes anterior al mes actual. \\
\hline
3.2 & El sistema obtiene el valor bruto total sumando los importes de todos los
movimientos obtenidos en 3.1. \\
\hline
3.3 & El sistema obtiene las comisiones cobradas al comercio calculando el 5\%
del total bruto obtenido en 3.2. \\
\hline
3.4 & El sistema obtiene el importe total a pagar al comercio restando del
valor bruto de 3.2 el valor de las comisiones de 3.3. \\
\hline
3.5 & El sistema emite un informe impreso que detalla los movimientos obtenidos
en 3.1, el importe bruto calculado en 3.2, el valor de las comisiones calculado
en 3.3 y el total a pagar al comercio calculado en 3.4. \\
\hline
4 & Fin del caso de uso. \\

\hline
\end{tabularx}

